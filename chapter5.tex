\documentclass{standalone}
% Load any packages needed for this document
\begin{document}
\section{Chapter 5 Hardware for VR}
complete VR installation consists of three main parts:
\begin{itemize}
\item User field: consists of various persons and/or application fields
\item Simulation field: consists of efficient computer(calculates presentation from the virtual environment) and also the feedback from the user
\item Interaction field:interaction field consists of the human-computer interface, which is used for the analog-in and output of the VR data 
\end{itemize}
\subsection{Visual In- and Output Devices}
\begin{itemize}
\item eye is the most important perception channel
\item Active Displays: generate the light for illumination(CRT, LED, VFD, plasma display)
\item use ambient light or background illumination(LCD)
\end{itemize}
\subsubsection{Display Technologies}
\subsubsection*{Cathod Ray Tube(CRT)}
\begin{itemize}
\item Developed by Ferdinand Braun at 1909
\item For multiple colors, penetron tube and mask tube is used
\begin{itemize}
\item Penetron Tube: works in the same way as cathod ray tube but different floresence colors are stacked at thte inside of the tube. In order to activate different layers, the electrons must have different kinetic energies.
\item Mask Tube: florescence colors are arranged in small triples or stripes(dots are visible if enlarged)
\end{itemize}
\item hole-mask or a slot-mask is used infront of the phosphoric layer which eliminates all unfocused electrons due to its positive charge
\item limitation of resolution is not given by cathod ray-tube but by the image is digitally processed
\begin{itemize}
\item vector technology: pros(no discretization, small amount of memory transformation in hardware) cons(only polygons, no surfaces, only low complexity otherwise image flicker)
\item grid technology: pros(surfaces can be displayed, simple and cheap, complex geometry possible) cons(visible steps, framebuffer needed, scan conversion needed) : ongoing technical development completely replaced by grid technology.
\end{itemize}
\end{itemize}
\subsubsection*{LC Displays}
\begin{itemize}
\item LC: Liquid Crystal in fluid state behaves like a fluid but also show some crystalline properties at the same time.
\item Three different states: nematic smectic, cholesteric
\begin{itemize}
\item Nematic: in nematic state all 
\end{itemize}
\end{itemize}
\subsubsection*{Ferroelectric Liquid Crystal Displays(FLCD)}
\subsubsection*{LED-Screens}
\subsubsection*{Plasma Displays(PDP)}
\subsubsection*{OLED Techonology}
\subsubsection*{Field Emission Display}
\subsubsection*{Electronic Paper}
\subsubsection*{Comparison between the Different Technologies}

\subsubsection{Projectors}
\subsubsection{Stereo Display Technologies}









\subsection{•}
\end{document}