\documentclass{standalone}
\begin{document}
\section{Chapter 2 Human Factors}
\begin{itemize}
	\item The goal is to stimulate all perception channels in such a way that the user feels completely immersed in the virtual environment and accepts it as real.
	\item visual, accoustic thermal, haptic and olfactory perception($80 \%$ of the overall information is perceived via the visual channel)
	\item Human Perception
	\begin{itemize}
		\item Perception
		\begin{itemize}
			\item Sensous physiology: the perception of stimuli, performed by sensory cells or by sensory organs.
			\item Psychology : Process of sensuous perception of an object without any conscious identification of the perceived object.
		\end{itemize}
		\item Psychology
		\begin{itemize}
			\item Psychology : Collective name for all processes or structures that are involved in the cognition process. e.g. imagination, estimation, mnemory, rememberance, learning. etc
		\end{itemize}
	\end{itemize}

	\item Four elementary attributes of stimulus
	\begin{itemize}
		\item modality
		\item intensity
		\item duration
		\item location
	\end{itemize}
	\item only the sensory part of the cognition process can be addressed by devices of the virtual environment. 
\end{itemize}

\subsection{The Human Eye}
\subsubsection{Viewing Angle}
\begin{itemize}
	\item Field of view characterized by: optical setup of the eye / the position of the eye in the face
	\item although field of view is very large perception of sharp images only possible in much smaller area
	\item viewing cones of both eyes overlap in a reange of approximately 120 degrees
\end{itemize}
\subsubsection{Temporal Resolution}
\begin{itemize}
	\item pupil can adapt to changing light conditions with a maximum frequency of 4 Hz	
	\item flicker consolidation frequency: maximum possible frequency without noticing any flickering
	\item perception of flickering not only depends on brightness but also the size of field of view and the location of light source it the FOV
\end{itemize}
\subsubsection{Accommodation and Convergence}
\begin{itemize}
	\item eye focuses onto shortest possible distance until object is visible: needs edges, patterns or contrours to focus
	\item depth estimation up to 10 meters is made with stereo scopic vision
\end{itemize}
\subsubsection{The Eye - Principle Set-up}
\subsubsection{B/W Perception, Color Perception}
\begin{itemize}
	\item color perception is done by a single eye
	\item color is perceived using special receptors on the retina
	\begin{itemize}
		\item sticks : only measure the intensity of the incoming light(Black and White). used at night but easily saturated.
		\item uvulas : used in daylight(light at night are not sufficient to work for uvulas), different uvulas for each color : (blue-sensitive uvulas, green-sensitive uvulas, red sensitive uvulas)
	\end{itemize}
	\item colors are coded into three basic colors, then two difference channels (red - green), (blue - yellow). addition of all colors result in the non-colored brightness.
	\item color models
	\begin{itemize}
		\item CIE color models
		\item YUV color model		
		\item RGB color model
		\item CMY color model
		\item HSV color model
	\end{itemize}
\end{itemize}

\subsubsection{Three-dimensional Viewing}
\begin{itemize}
	\item Spatial Viewing
	\begin{itemize}
		\item Size of the Image on the Retina
		\item Resoultion of the Perceived Image
		\item Overlap
		\item perspective
		\item shaded objects and shadows
		\item textures
		\item motion parallax
	\end{itemize}
	\item Perception Rules Depending on geometry
	\begin{itemize}
		\item Rules of proximity
		\item rule of similarity
		\item identity versus proximity
		\item rule of harmonic continuation
		\item rule of closed lines
		\item rule of symmetry
		\item the principle of harmonic shape
		\item contours
	\end{itemize}
	\item Allocation Problems
	\item Stereoscopic viewing
\end{itemize}
\subsubsection{Conflicts between Virtual Reality and the Physiological and Psychological Visual Perception}
\subsection{The Human Ear}
\begin{itemize}
	\item Functionality of hearing
	\item Sound signals and sound sources
	\item Location of sound sources
	\item sensitivity and loudness(very important psychological factor)
\end{itemize}
\subsubsection{Anatomical Set-up of the Ear}
\begin{itemize}
	\item Human ear can be seprated into three, anatomically different organs
		\begin{itemize}
			\item outer ear
			\item middle ear
			\item inner ear
		\end{itemize}
	\item sensitivity of the human ear is frequency dependent
	\item accoustic pressure level $L_p$ is a quantity that can be directly measured but not directly correspond to the subjective impression of loudness
	\begin{equation}
	L_p = 20 log \frac{p}{p_0} [dB]
	\end{equation}
	\item The complete field between the pain threshold and the accoustic threshold is called the perception area
\end{itemize}
\subsubsection{Spatial Hearing}
Spatial hearing desribes the accoustical determination of a sound source's position in a room. Two princilples:
\begin{itemize}
	\item monaural hearing
	\begin{itemize}
		\item Using one ear only: very rough localization($\pm 20 \deg deviation from correct localization$) : does not play important role
	\end{itemize}
	\item binaural hearing
	\begin{itemize}
		\item difference in intensity: only possible if wavelength of the sound is small compared to the size of the head
		\item difference in runtime: brain capable of detect runtime differences of approximately 30us
		\item can calculate the run time difference:
		$$
		\Delta x = d sin \alpha \\
		\Delta t = \frac{\Delta x}{c} = \frac{d}{c} sin \alpha \quad c = 340m/s
		$$
	\end{itemize}
	\item Sound and Tone
		\begin{itemize}
			\item A sound sensation, which has a periodic oscillation but a non-sinusoidal waveform can be generated by a superposition of multiple sinewaves with different amplitudes and frequencies(frequencies have to be in an integral relation ship with each other). All sound sensations that fulfill the requirement are also named a tone
		\end{itemize}
	\item Allocation Problems of the ear
			\begin{itemize}
			\item Shephard effect: simulates increasing melody to the user although the pitch stays constant all the time
		\end{itemize}
\end{itemize}

\subsection{The Haptic Channel}
\subsubsection{The Human Information Flow}
\begin{itemize}
	\item the entirety of all perceptions through the sense of touch
	\item about 80 \% of all information is perceived by the visual sense, other 15 \% by the auditory sense. The remaining 5 \% are fr the haptic and olfactoric sense. 
\end{itemize}
\subsubsection{The Haptic Perception}
\begin{itemize}
	\item The haptic takes place all over the body, thus it is impossible to generate a complete simulation in the haptic field: The human hand is the most relevant element in the field of haptics
	\item well known input device: mouse and keyboard provide haptic feedback addition to acoustic feedback.
	\item required for teleoperation
	\item Using the had to perform a manipulation
	\begin{itemize}
		\item Contact Phase
		\item Grasping Phase
		- Grasping for increased power / Grasping with increased dexterity
		\item Touching Phase
	\end{itemize}
	\item kinesthetic sensation is the perception of acceleration forces onto the body
	\item Skin registers pressure, touch, vibration (=sense of touch), temperature and pain. This surface sensibility (together with the depth sensibility (muscle, joint and string receptors)) is also called somatovisceral sensibility
	\begin{itemize}
		\item Meissner cells and hair receptors: detect the touch.(Not the intensity is important(bending of the hair) but the speed by the sensation changes.
		\item Pacini cells: specialized to detect vibrations
	\end{itemize}
	\item receptors can be divided into slowly adapting and rapdily apdapting mechanoreceptors
	\begin{itemize}
		\item rapidly adapting receptors respond to onset and often also termination but not throughout during stimulus-> sensorial adaptation
	\end{itemize}
	\item receptors can be classified by spatial resolution and temporal resolution
	\begin{itemize}
		\item intensity receptors: P-receptors
		\item speed receptors: D-receptors
		\item PD receptors are a mixture, which measure for example the position of the joints
	\end{itemize}
	\item Thermo receptors exist for a temperature range below 36 degrees
\end{itemize}
\subsubsection{Depth Sensibility; Distension Reflex}
\begin{itemize}
	\item proprioreceptors
\end{itemize}
\subsection{The collaboration of all Senses}
\begin{itemize}
	\item Information flow during action
	\item McGurk effect
\end{itemize}
\end{document}