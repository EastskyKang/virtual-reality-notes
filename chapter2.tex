\documentclass{standalone}
\begin{document}
\section{Chapter 2 Human Factors}
\begin{itemize}
	\item The goal is to stimulate all perception channels in such a way that the user feels completely immersed in the virtual environment and accepts it as real.
	\item visual, accoustic thermal, haptic and olfactory perception($80 \%$ of the overall information is perceived via the visual channel)
	\item Perception
		\begin{itemize}
		\item Sensous physiology: the perception of stimuli, performed by sensory cells or by sensory organs.
		\item Psychology : Process of sensuous perception of an object without any conscious identification of the perceived object.
		\end{itemize}
		\item Psychology
			\begin{itemize}
				\item Sensous physiology: the perception of stimuli, performed by sensory cells or by sensory organs.
				\item Psychology : Collective name for all processes or structures that are involved in the cognition process. e.g. imagination, estimation, mnemory, rememberance, learning. etc
	\end{itemize}
\end{itemize}

\subsection{The Human Eye}

\subsubsection*{Viewing Angle}
\subsubsection{Temporal Resolution}

\subsubsection{Accommodation and Convergence}

\subsubsection{B/W Perception, Color Perception}
\subsubsection{Three-dimensional Viewing}

\textbf{Spatial Viewing}
\textbf{Size of the Image on the Retina}
\textbf{Resoultion of the Perceived Image}
\textbf{Overlap}
\end{document}