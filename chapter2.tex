\documentclass{standalone}
\begin{document}
\section{Chapter 2 Human Factors}
\begin{itemize}
	\item The goal is to stimulate all perception channels in such a way that the user feels completely immersed in the virtual environment and accepts it as real.
	\item visual, accoustic thermal, haptic and olfactory perception($80 \%$ of the overall information is perceived via the visual channel)
	\item Human Perception
	\begin{itemize}
	\item Perception
		\begin{itemize}
			\item Sensous physiology: the perception of stimuli, performed by sensory cells or by sensory organs.
			\item Psychology : Process of sensuous perception of an object without any conscious identification of the perceived object.
		\end{itemize}
	\item Psychology
		\begin{itemize}
			\item Psychology : Collective name for all processes or structures that are involved in the cognition process. e.g. imagination, estimation, mnemory, rememberance, learning. etc
		\end{itemize}
    \end{itemize}
    
    \item Four elementary attributes of stimulus
    \begin{itemize}
    		\item modality
    		\item intensity
    		\item duration
    		\item location
    \end{itemize}
    \item only the sensory part of the cognition process can be addressed by devices of the virtual environment. 
\end{itemize}

\subsection{The Human Eye}
\subsubsection{Viewing Angle}
\begin{itemize}
	\item Field of view characterized by: optical setup of the eye / the position of the eye in the face
	\item although field of view is very large perception of sharp images only possible in much smaller area
	\item viewing cones of both eyes overlap in a reange of approximately 120 degrees
\end{itemize}
\subsubsection{Temporal Resolution}
\begin{itemize}
	\item pupil can adapt to changing light conditions with a maximum frequency of 4 Hz	
	\item flicker consolidation frequency: maximum possible frequency without noticing any flickering
	\item perception of flickering not only depends on brightness but also the size of field of view and the location of light source it the FOV
\end{itemize}
\subsubsection{Accommodation and Convergence}
\begin{itemize}
	\item eye focuses onto shortest possible distance until object is visible: needs edges, patterns or contrours to focus
	\item depth estimation up to 10 meters is made with stereo scopic vision
\end{itemize}
\subsubsection{The Eye - Principle Set-up}
\subsubsection{B/W Perception, Color Perception}
\begin{itemize}
\item color perception is done by a single eye
\item color is perceived using special receptors on the retina
	\begin{itemize}
		\item sticks : only measure the intensity of the incoming light(Black and White). used at night but easily saturated.
		\item uvulas : used in daylight(light at night are not sufficient to work for uvulas), different uvulas for each color : (blue-sensitive uvulas, green-sensitive uvulas, red sensitive uvulas)
	\end{itemize}
	\item colors are coded into three basic colors, then two difference channels (red - green), (blue - yellow). addition of all colors result in the non-colored brightness.
	\item color models
	\begin{itemize}
		\item CIE color models
		\item YUV color model		
		\item RGB color model
		\item CMY color model
		\item HSV color model
	\end{itemize}
\end{itemize}

\subsubsection{Three-dimensional Viewing}
\begin{itemize}
\item Spatial Viewing
	\begin{itemize}
		\item Size of the Image on the Retina
		\item Resoultion of the Perceived Image
		\item Overlap
		\item perspective
		\item shaded objects and shadows
		\item textures
		\item motion parallax
\end{itemize}
\item Perception Rules Depending on geometry
	\begin{itemize}
		\item Rules of proximity
		\item rule of similarity
		\item identity versus proximity
		\item rule of harmonic continuation
		\item rule of closed lines
		\item rule of symmetry
		\item the principle of harmonic shape
		\item contours
	\end{itemize}
\item Allocation Problems
\item Stereoscopic viewing
\end{itemize}
\subsubsection{Conflicts between Virtual Reality and the Physiological and Psychological Visual Perception}
\subsection{The Human Ear}
\subsubsection{Anatomical Set-up of the Ear}
\subsubsection{Spatial Hearing}
\subsection{The Haptic Channel}

\end{document}