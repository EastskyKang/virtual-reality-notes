\documentclass[12pt]{article}

\usepackage{sbc-template}

\usepackage{graphicx,url}

\usepackage[english]{babel}   
%\usepackage[latin1]{inputenc}  
\usepackage[utf8]{inputenc}  
% UTF-8 encoding is recommended by ShareLaTex
\usepackage{verbatim}
\usepackage{listings}
\usepackage{xcolor}
\usepackage{graphicx}
\usepackage{listings}
\usepackage[]{algorithm2e}
\usepackage{color}
\usepackage{amsmath}
\usepackage{amsfonts}
\usepackage{multirow}
\RestyleAlgo{ruled}
\usepackage{fancybox}
\usepackage{siunitx}
\usepackage{fixltx2e}
\usepackage{standalone}

\graphicspath{ {Figures/} }

\newcommand\norm[1]{\left\lVert#1\right\rVert}
\definecolor{verde}{rgb}{0,0.5,0}
\DeclareMathAlphabet{\mathpzc}{OT1}{pzc}{m}{it}
%para customizar o código (ver https://en.wikibooks.org/wiki/LaTeX/Source_Code_Listings)
\lstset{language=C, %defina a linguagem usada no trabalho
              belowcaptionskip=1\baselineskip,
                breaklines=true,
                frame=false,
                xleftmargin=\parindent,
                showstringspaces=false,
                basicstyle=\footnotesize\ttfamily,
                keywordstyle=\bfseries\color{green!40!black},
                commentstyle=\itshape\color{purple!40!black},
                identifierstyle=\color{blue},
                stringstyle=\color{orange},
                numbers=left,
            }

\sloppy

\title{Virtual Reality 1}

\author{Dongho Kang, Jaeyoung Lim, Soomin Lee, Jaeryeong Choi}

\address{kangd@ethz.ch}

\begin{document} 

\maketitle

\section{Chapter 1: Introduction into Virtual Reality}


\section{Chapter 2}
\documentclass{standalone}
\begin{document}
\section{Chapter 2 Human Factors}
\begin{itemize}
	\item The goal is to stimulate all perception channels in such a way that the user feels completely immersed in the virtual environment and accepts it as real.
	\item visual, accoustic thermal, haptic and olfactory perception($80 \%$ of the overall information is perceived via the visual channel)
	\item Human Perception
	\begin{itemize}
		\item Perception
		\begin{itemize}
			\item Sensous physiology: the perception of stimuli, performed by sensory cells or by sensory organs.
			\item Psychology : Process of sensuous perception of an object without any conscious identification of the perceived object.
		\end{itemize}
		\item Psychology
		\begin{itemize}
			\item Psychology : Collective name for all processes or structures that are involved in the cognition process. e.g. imagination, estimation, mnemory, rememberance, learning. etc
		\end{itemize}
	\end{itemize}
	\item Four elementary attributes of stimulus
	\begin{itemize}
		\item modality: quality of a stimulus(the type of physical energy that is responsible for the stimulus)
		\item intensity: The strength of a stimulus defines the intensity of a sensation(below a specific threshold, the stimulus is not detected
		\item duration: relationship between the intensity of the stimulus and the duration of the sensation defines the perceived intensity
		\item location: The ability to locate the position of a stimulus and the ability to distinguish between two spatially close stimuli are important measures of the awareness of the spatial distribution of a sensory experience
	\end{itemize}
	\item only the sensory part of the cognition process can be addressed by devices of the virtual environment. (Figure)
\end{itemize}
\begin{figure}[H]
\centering
\includegraphics[width = 0.7\linewidth]{Figures/2_3.png}
\end{figure}
\subsection{The Human Eye}
\subsubsection{Viewing Angle}
\begin{itemize}
	\item Field of view characterized by: optical setup of the eye / the position of the eye in the face
	\item although field of view is very large perception of sharp images only possible in much smaller area
	\item viewing cones of both eyes overlap in a reange of approximately 120 degrees
	\item Four opening angles of the field of view
\begin{table}[H]
\centering
\begin{tabular}{|l|c|c|r|}
\hline
 nasal & temporal & superior & inferior \\ \hline
60 deg & 100 deg & 60 deg & 70 deg \\ \hline
\end{tabular}
\end{table}
\end{itemize}
\subsubsection{Temporal Resolution}
\begin{itemize}
	\item pupil can adapt to changing light conditions with a maximum frequency of 4 Hz	
	\item flicker consolidation frequency: maximum possible frequency without noticing any flickering
	\item perception of flickering not only depends on brightness but also the size of field of view and the location of light source it the FOV
\end{itemize}
\subsubsection{Accommodation and Convergence}
\begin{itemize}
	\item eye focuses onto shortest possible distance until object is visible: needs edges, patterns or contrours to focus: In compelte darkness, the eye cannot focus anymore
	\item results in near sightedness when looking through a optical device
	\item depth estimation up to 10 meters is made with stereoscopic vision $$ tan \frac{\epsilon}{2} = \frac{a}{2} \frac{1}{e}$$
\end{itemize}
\subsubsection{The Eye - Principle Set-up}
\begin{itemize}
	\begin{figure}[H]
	\centering
	\includegraphics[width = 0.7\linewidth]{Figures/2_12.png}
	\end{figure}
\item eyeball has almost the shape of a sphere. Its skin consists of three layers: the outer sclera, athe arteries, and the retina
\item Light goes through the eyeball onto the retina
\item Retina consists of light receptors called uvulas and sticks which convert incoming light into electrical impulses
\item lens
\begin{itemize}
\item lens focuses the object onto the retina by adapting their focal length to the distance of the object
\item lens is fixed by fibers to the ciliar muscle. Activating or deactivating the muscle can change the geometry of the lens and thus the refraction
\end{itemize}
\end{itemize}
\subsubsection{B/W Perception, Color Perception}
\begin{itemize}
	\item light models: wave-model is more sufficient regarding the light as a electromagnetic radiation with a wavelength between 380nm and 780 nm
	\item color perception is done by a single eye
	\item color is perceived using special receptors on the retina
	\begin{itemize}
		\item sticks : only measure the intensity of the incoming light(Black and White). used at night but easily saturated.
		\item uvulas : used in daylight(light at night are not sufficient to work for uvulas), different uvulas for each color : (blue-sensitive uvulas 4 \% at 430nm , green-sensitive uvulas 32 \% at 530 nm, red sensitive uvulas 64 \% at 560 nm)
	\end{itemize}
	\item colors are coded into three basic colors, then two difference channels (red - green), (blue - yellow). addition of all colors result in the non-colored brightness.
		\begin{figure}[H]
			\centering
			\includegraphics[width = 0.5\linewidth]{Figures/2_16.png}
		\end{figure}
\item During daylight: different kind of uvulas capture the basic colors red, green and blue while the sticks only measure brightness
\item During darkness: uvulas do not provide any color information, but only black(black does give any information to the brain?), small amount of light is sufficient to activate the sticks without saturating them
\item hue, saturation, brightness imitate the human color perception
\item there is also a contrast amplification in oerder to increase the perceived sharpness of the image
\item focal length depends on the color(Chromosteropsis: if more colors exist at the same distance, the eye can only focus on them, if red, blue exist the eye will focus on the red field due to the larger amount of red sensitive receptors)
\end{itemize}
\subsubsection*{Color Models}
\begin{itemize}
		\item CIE color model(Commission Internationale del'Eclairage): based on the typical sensitivity of the different uvulas(perception oriented color model)
		\item YUV color model: Takes into account the high green sensitivity into account, while sensitivity to red and blue is significantly lower(perception oriented color model)
		$$ C_b = B - Y color difference signal 1 \\
		C_r = R - Y color difference signal 2 \\
		Y brightness (luminance) $$
		\item RGB color model: does not take the physiology of the human eye into account(technical color model), basic colors are scaled to 1 and define a coordinate system
		\item CMY color model: is used by objects that do not emit light themselves, color used by printers
		\item HSV color model(also HSB): Defined by hue, saturation, value. In this color model, colors are arranged in a circle around the vertical axis
\end{itemize}
\subsubsection{Three-dimensional Viewing}
\subsubsection*{Spatial Viewing}
\begin{itemize}
\item spatial viewing can be performed with only one eye and thus is also called monocular viewing(only physiological aspects)
	\begin{itemize}
		\item Size of the Image on the Retina: If the real size is known, the brain compares the perceived size with the real size stored in the brain and calculates information of the position and the distance of the object
		\item Resoultion of the Perceived Image: Objects with blurred surface appear far away
		\item Overlap: If one object is in front of the other it has to be closer to the spectator than the object
		\item perspective: Far objects appear smaller than close ones
		\item shaded objects and shadows: I the light source is know, the object that causes the shadow appears closer to the light source
		\item textures: A spatial effect also appears when mapping textures perspectively onto an object. Wrong selection of textures could completely camouflage the geometry or pretend a wrong geometry
		\item motion parallax: the farther away an object is, the slower it moves on the retina when moving relatively to the user
	\end{itemize}
	\item Perception Rules Depending on geometry
	\begin{itemize}
		\item Rules of proximity: Spatially or temporally neighboring elements are perceived to belong together and to be part of the same element
		\begin{figure}[H]
			\centering
			\includegraphics[width = 0.5\linewidth]{Figures/2_39.png}
		\end{figure}
		\item rule of similarity/rule of identity: Similar or identical objects appear coherently 
		\begin{figure}[H]
			\centering
			\includegraphics[width = 0.5\linewidth]{Figures/2_40.png}
		\end{figure}
		\item identity versus proximity: effects can be amplified or reduced
		\begin{figure}[H]
			\centering
			\includegraphics[width = 0.5\linewidth]{Figures/2_41.png}
		\end{figure}
		\item rule of harmonic continuation: elements that are spatially or temporally arranged in a simple harmonic or well defined order appear coherently and thus belong to the same geometric figure
		\begin{figure}[H]
			\centering
			\includegraphics[width = 0.5\linewidth]{Figures/2_42.png}
		\end{figure}
		\item rule of closed lines: Contours, which are not completely closed, will be automatically closed during the perception process
			\begin{figure}[H]
				\centering
				\includegraphics[width = 0.5\linewidth]{Figures/2_44.png}
				\end{figure}
		\item rule of symmetry: if none of the mentioned rules could be applied, the space between symmetric contours shapes a figure, rather than the space between asymmetric contours
			\begin{figure}[H]
				\centering
				\includegraphics[width = 0.5\linewidth]{Figures/2_45.png}
			\end{figure}
		\item the principle of harmonic shape: The structure will be perceived, that has as many simple figures as possible
			\begin{figure}[H]
			\centering
			\includegraphics[width = 0.5\linewidth]{Figures/2_46.png}
			\end{figure}
		\item contours: 
			\begin{figure}[H]
			\centering
			\includegraphics[width = 0.7\linewidth]{Figures/2_47.png}
			\end{figure}
	\end{itemize}
\end{itemize}
\subsubsection*{Allocation Problems}
\begin{itemize}
	\item wrong perspective, brightness of objects given context, comparison with known patterns
	\item peripheral drift illusion: depends on difference in brightness or color
	\item Hermann grid: Different structures of receptive areas can be realized by individual interconnections
		\begin{figure}[H]
			\centering
			\includegraphics[width = 0.7\linewidth]{Figures/2_58.png}
		\end{figure}
\end{itemize}
\subsubsection*{Stereoscopic viewing}
\begin{itemize}
	\item The point is projected on non-corresponding areas on the retina on both eyes. This gives a spatcial impression
\end{itemize}
\subsubsection{Conflicts between Virtual Reality and the Physiological and Psychological Visual Perception}
\begin{itemize}
	\item Since all virtual objects are visible on the projection plane, the human eye does not have to perform an accommodation anymore, but only focuses once on the given distance between the user and the projection plane
	\item psychological apsects are also not fulfilled
\end{itemize}
\subsection{The Human Ear}
\begin{itemize}
	\item Functionality of hearing
	\item Sound signals and sound sources
	\item Location of sound sources
	\item sensitivity and loudness(very important psychological factor)
\end{itemize}
\subsubsection{Anatomical Set-up of the Ear}
\begin{itemize}
	\item Human ear can be seprated into three, anatomically different organs
		\begin{itemize}
			\item outer ear: auricle and the auditory canal belong to the outer ear
			\item middle ear: a mechanical impedance converter is situated between the eardrum and the inner ear, consisting of three bones-hammer, anvil, stirrup
			\item inner ear: Consists of a spiral pipe, which gets thinner and is filled with fluid, the medium for acoustic wave dissemination is the air, while in the inner ear a fluid(lymph) is ued
		\end{itemize}
	\item sensitivity of the human ear is frequency dependent
	\item accoustic pressure level $L_p$ is a quantity that can be directly measured but not directly correspond to the subjective impression of loudness
	\begin{equation}
	L_p = 20 log \frac{p}{p_0} [dB]
	\end{equation}
	\item The complete field between the pain threshold and the accoustic threshold is called the perception area
\end{itemize}
\subsubsection{Spatial Hearing}
Spatial hearing desribes the accoustical determination of a sound source's position in a room. Two princilples:
\begin{itemize}
	\item monaural hearing
	\begin{itemize}
		\item Using one ear only: very rough localization($\pm 20 \deg deviation from correct localization$) : does not play important role
	\end{itemize}
	\item binaural hearing
	\begin{itemize}
		\item difference in intensity: only possible if wavelength of the sound is small compared to the size of the head
		\item difference in runtime: brain capable of detect runtime differences of approximately 30us
		\item can calculate the run time difference:
		$$
		\Delta x = d sin \alpha \\
		\Delta t = \frac{\Delta x}{c} = \frac{d}{c} sin \alpha \quad c = 340m/s
		$$
	\end{itemize}
	\item Sound and Tone
		\begin{itemize}
			\item A sound sensation, which has a periodic oscillation but a non-sinusoidal waveform can be generated by a superposition of multiple sinewaves with different amplitudes and frequencies(frequencies have to be in an integral relation ship with each other). All sound sensations that fulfill the requirement are also named a tone
		\end{itemize}
	\item Allocation Problems of the ear
			\begin{itemize}
			\item Shephard effect: simulates increasing melody to the user although the pitch stays constant all the time
		\end{itemize}
\end{itemize}
\subsection{The Haptic Channel}
\subsubsection{The Human Information Flow}
\begin{itemize}
	\item the entirety of all perceptions through the sense of touch
	\item about 80 \% of all information is perceived by the visual sense, other 15 \% by the auditory sense. The remaining 5 \% are fr the haptic and olfactoric sense. 
\end{itemize}
\subsubsection{The Haptic Perception}
\begin{itemize}
	\item The haptic takes place all over the body, thus it is impossible to generate a complete simulation in the haptic field: The human hand is the most relevant element in the field of haptics
	\item well known input device: mouse and keyboard provide haptic feedback addition to acoustic feedback.
	\item required for teleoperation
	\item Using the had to perform a manipulation
	\begin{itemize}
		\item Contact Phase: Describes the first contact of the fingers with an object(felt 200 ms after contact)
		\item Grasping Phase: has the largest flexibility and interactability
		- Grasping for increased power / Grasping with increased dexterity
		\item Touching Phase: For characterizing the properties of objects, different hand-object interactions are needed
	\end{itemize}
	\item Grasping can be seperated with thwo fields
	\begin{itemize}
		\item grasping with increased power
		\item grasping with increased dexterity
	\end{itemize}
	\item Haptic desribes the influence of foces of any kind onto the human body(tactile propriorceptive, kinesthetic)
	\begin{itemize}
		\item tactile: using perception cells in the skin(pressure, temperature, vibration)
		\item propriorceptive: Influence of force caused by the weight of the object onto the sensors in the musculature
		\item kinesthetic sensation is the perception of acceleration forces onto the body 
	\end{itemize}
\end{itemize}
\subsubsection*{Basics on the Physiology of the Senses in the Skin}
\begin{itemize}
	\item Skin registers pressure, touch, vibration (=sense of touch), temperature and pain. This surface sensibility (together with the depth sensibility (muscle, joint and string receptors)) is also called somatovisceral sensibility
	\begin{itemize}
		\item Meissner cells and hair receptors: detect the touch.(Not the intensity is important(bending of the hair) but the speed by the sensation changes.
		\item Pacini cells: specialized to detect vibrations
	\end{itemize}
	\item receptors can be divided into slowly adapting and rapdily apdapting mechanoreceptors
	\begin{itemize}
		\item rapidly adapting receptors respond to onset and often also termination but not throughout during stimulus-> sensorial adaptation
	\end{itemize}
	\item receptors can be classified by spatial resolution and temporal resolution
	\begin{itemize}
		\item intensity receptors: P-receptors
		\item speed receptors: D-receptors
		\item PD receptors are a mixture, which measure for example the position of the joints
	\end{itemize}
	\item Thermo receptors exist for a temperature range below 36 degrees
\end{itemize}
\subsubsection{Depth Sensibility; Distension Reflex}
\begin{itemize}
	\item proprioreceptors: for measuring the position of the joints, the length of the muscles etc(depth sensibility) so-called proprioceptors exist
\end{itemize}
\subsection{The collaboration of all Senses}
\begin{itemize}
	\item conscious and unconscious perception can be described using the so-called inner model
	\begin{itemize}
		\item Describes the context between the performed action and the expected sensor information
		\item the inner model contains the expected values of the receptors for the performed action
	\end{itemize}
	\item Pushing a button example
	\begin{figure}[H]
			\centering
			\includegraphics[width = 0.5\linewidth]{Figures/2_82.png}
	\end{figure}
	\item McGurk effect: If acoustic and visual stimuli are not correlated
\end{itemize}
\end{document}
\newpage
\section{Chapter 3: Introduction into Computer Graphics}

\subsection{Introduction}

\begin{itemize}
	\item Quick development in computer graphics: from an expensive toy to an attractive research field
		\begin{itemize}
			\item main reason: human receives most of the information through the visual perception channel
			\end{itemize}
	\item Rapid development concerning hardware, software, and applications
		\begin{itemize}
			\item graphics can be easily accessed and understood by different cultures/people/languages (high synergy among people)
		\end{itemize}
\end{itemize}

\subsection{Why do we need Computer Graphics?}

\begin{itemize}
	\item The representation of texts is also a special form of graphics
	\item Human being seems to have a better access to images and pictures than letters and numbers
		\begin{itemize}
			\item e.g. complex simulation results by visualization, logo of a company
		\end{itemize}
\end{itemize}

\subsubsection*{Computer Graphics and Picture Recognition}

\includegraphics[scale=0.6]{3_1}

\begin{itemize}
	\item When a human perceives the surrounding world, the brain generates an abstract "data model"
	\item Computer Vision: extract relevant data out of existing graphics or images
	\item Computer Graphics: display the data
\end{itemize}

\subsection{Applications of Computer Graphics}

\subsubsection*{CAD}

\begin{itemize}
	\item Construction (Functional Design)
	\begin{itemize}
		\item the requirements to an assembly part are well defined by function and dimension
		\item all drawings on a CAD system are a complex relationship between dimensions, constraints, and material properties, which describe a part very well
	\end{itemize}
	\item Design (Aesthetic Design)
	\begin{itemize}
		\item do not focus on the dimension of a part, but factors like ergonomics or aesthetics
		\item many suppliers of design software try to combine the relatively free aesthetic design field with the very compulsory functional design, but hasn't been successful due to difficult exchange of data between both worlds
	\end{itemize}
\end{itemize}

\subsubsection*{Gaming}
\begin{itemize}
	\item Caused a significance increase of graphic performance in the private computer sector
	\item Cheap components and fast, high- quality software were developed due to the growing demands of players and the competition of the industry
\end{itemize}

\subsubsection*{Visualization}

\begin{itemize}
	\item Visualization is the ancestor of 3D computer graphics
	\item \textbf {Definition: Visualization addresses special properties of human perception to visualize and to represent information, which is extracted from large amount of data}
	\item Visualization intends to:
	\begin{itemize}
		\item display structures, models, trends, anomalies, and relationships
		\item give an overview of large amounts of data
		\item give support by means of an easy modification of parameters when searching for interesting regions in a large data field
	\end{itemize}
	These points can be summarized in the mantra of Ben-Shneiderman:
	\begin{center}
	\textbf{overview, zoom-in, filtering, details on request}
	\end{center}
	Thus, a good visualization has the following properties:
	\begin{itemize}
		\item it prevents misinterpretation and ambiguity
		\item it optimizes the perception of subtle properties
		\item it allows displaying more data at a time
	\end{itemize}
	\item Typical data sets which need to be visualized very often:
	\begin{itemize}
		\item MRI (the density can be visualized by different colors and transparencies)
		\item CFD (the location and the direction of movement of points with 6 DOF, color and particles, size or shape can display the information)
		\item Financial Data (display in a diagram, so that a user can see correlations between variables)
		\item CAD (3D data with additional information for edges, corners, surfaces, and surface properties. Complex data structures are used because the data is used not only for visualization but also for other processes. Complexity of the models strongly depends on the display quality)
		\item Statistical data sets (the correlation between different parameters only can be seen through visualization)
	\end{itemize}
	\item Different kinds of visualization in the following table. The different categories do overlap in many cases, "Illustration" has an exceptional position since it can be used everywhere
\end{itemize}

\includegraphics[scale=0.7]{3_8}

\begin{itemize}
	\item Data Visualization
		\begin{itemize}
			\item The most common way would be Excel-like presentation (e.g. pancake diagram, 3D graphs)
		\end{itemize}
	\item Cybernetic Visualization (Simulation)
		\begin{itemize}
			\item The main interest in the visualization of a cybernetic application: speed and possible detection of errors and irregular behavior of the simulated system
			\item The classical way: showing data in a 2D graph as a time-dependent value
		\end{itemize}
	\item Multi-dimensional Visualization
		\begin{itemize}
			\item when much data has to be displayed simultaneously within one image
			\item every point in the space can represent 3 independent values + color, line thickness, transparency, different shapes, etc.
			\item other ways exist but a compromise has to be made concerning the simplicity (e.g. switch between different data visualization planes on any given point of a curve or shape)
		\end{itemize}
	\item Statistics
		\begin{itemize}
			\item typically statistics contain many different values for one single measurement.
			\item the visualization's task is showing possible correlations among the different values
			\item only valid for the given constraints
		\end{itemize}
\end{itemize}

\subsubsection*{Illustration}

\begin{itemize}
	\item Illustrations are 2D add-ons to a 3D object. They contain much additional information about the 3D representation.
	\item GUIs
		\begin{itemize}
			\item the most important means of communication between the user and the computer
			\item the most important input device is the mouse
			\item basic idea of GUI: pointing on an object is one of the easiest gestures of a human being
			\item within the GUI an icon is a small image, which is a representation for a program sequence or for the content of a defined type (they made working with a computer much faster and more comfortable because people can just 'click' on them instead of entering complex instructions)
		\end{itemize}
	\item Fonts
		\begin{itemize}
			\item within the visualization, text is used to assign abstract contents like titles, classifications, or dimensions to a graphical object
			\item Typesets and fonts are the medium of text (based on typography and applications)
		\end{itemize}
	\item Layout
		\begin{itemize}
			\item By placing graphical elements on a surface, the underlying information is structured more clearly and more detailed
			\item typical applications: windows-based desktops such as MacOS, Windows (e.g. word wrap)
			\item Another kind of layout is characterized by the given application (e.g. simulation of operation elements in a car- the ergonomic placement of the elements is essential for the overall handling of the vehicle)
		\end{itemize}
	\item Drawings
		\begin{itemize}
			\item very often used to visualize abstract situations
			\item e.g. organization charts or flowcharts, manuals
		\end{itemize}
	\item Instructions
		\begin{itemize}
			\item instructions consist of a mixture of text, graphical visualization, photos, etc.
			\item very often they visualize products in an abstract and simplified way
			\item e.g. service manuals, assembly manuals
		\end{itemize}
\end{itemize}

\subsection{Definition of 3D Graphics}

\begin{itemize}
	\item Definition of 3D graphics: \textbf{The field 3D graphics deals with the generation of 3D objects and their representation on a two-dimensional surface(e.g. a screen)}
	\item the main focus is on projection and visualization of a 3D space on a 2D surface of any kind
	\item example of 3D graphic devices: video camera, photo camera
	\item main characteristics of 3D graphics:
	\begin{itemize}
		\item data acquisition / data transfer / storage
		\item transformation / processing
		\item data display on 2D
		\item input and output
	\end{itemize}
	\item simple pipeline for data processing: \\
		$\rightarrow$ Definition of geometry as digital data \\
		$\rightarrow$ Processing and transformation of data \\
		$\rightarrow$ Transformation of data to 2D \\
		$\rightarrow$ Display of the results on a screen
\end{itemize}

\subsection{Rendering Pipeline}

\begin{itemize}
	\item The rendering pipeline is the basis for almost every application in computer graphics
	(can be slightly modified for special applications such as augmented reality)
	\item The simplest form of the rendering pipeline: \\
	$\rightarrow$ Processing and transformation of data \\
	$\rightarrow$ Transformation of data to 2D \\
	$\rightarrow$ Display of the results on a screen
	\item transformation / processing: modify the original data in such a way that it can be displayed well on a 2D surface. contains data modified by the user e.g. modified viewing angle onto the geometry
	\item transformation of data to 2D: all modified data has to be transformed onto the 2D surface. e.g. removal of hidden objects or surfaces, transferring a curved surface into triangles, or the integration of illumination models
	\item input/output: reduce the amount of data so that the image can be optimally displayed on the output device (due to limitation in resolution)
\end{itemize}

\subsection{Definition of Geometry in the Computer}

\begin{itemize}
	\item Algorithms for a geometric representation of curves and bodies were developed in order to model hulls, car bodies, and fuselages
	\item B\'ezier developed a CAD-system (further developed into the 'UNISURF' system later on), close approximation to geometry
	\item In order to define geometry, the information should be sufficient but not redundant
	\item It can be distinguished between volumetric and surface models
\end{itemize}

\subsubsection*{Discrete Definition of Surfaces}
\begin{itemize}
	\item Cloud of Points:
	\begin{itemize}
		\item the simplest model, every point of the cloud is defined by its coordinates
		\item The point cannot be rendered(no information on color or texture) and there is no connectivity among the points(separate parts cannot be selected in the model)
		\item 3D-scanner typically provides a cloud of points
		\item difficult to create a good surface out of the clouds of points because the acquired data is noisy and erroneous
	\end{itemize}
	\item Polygons, Tessellation:
	\begin{itemize}
		\item Tessellation is a method for defining or generating a mesh of geometric basic elements(polygons), which approximate a complex surface
		\item Polygon, especially triangle, is one of the most important basic elements of computer graphics
		\item "triangulation": the process in which a free-form surface is transformed into triangles (The mesh approximates the surface by triangles and reduces the complexity)
		\item additional data has to be considered when rendering (e.g. perpendicular of a surface)
	\end{itemize}
	\item Polystrips
	\begin{itemize}
		\item when a point is added to a triangle, a second triangle will be generated
		\item advantage: a large amount of surfaces can be generated by a very low amount of data
		\item stored in two tables: the point(vertex) table with all coordinates of the triangle's corners and the strip table with all coherent triangles
		\item details could be lost since any curve is approximated by straight lines of different lengths
		\item many times the triangulation is done at a very late stage in rendering because often working with mathematically exact representations of the geometry is required
	\end{itemize}
\end{itemize}

\subsubsection*{Mathematical Description of Geometries}

\begin{itemize}
	\item Typically, the surfaces of objects are represented by approximating functions
	\item A parametric representation is used for the approximation or interpolation of these surfaces (e.g. parametric polynomials - B\'ezier, B-Spline)
	\item Surface models are mainly used- drawback: there are no relationships among the individual parts of the surface (The individual surfaces are not correlated with an object and thus the shape of an object is a question of interpretation)
	\item possibilities for the mathematical description of curves and surfaces: explicit, implicit, parametric
	\begin{itemize}
		\item explicit: only one value for y is associated with the variable x\\
		thus problems arise for closed curves or objects\\
		it is not invariant to rotations\\
		only a few geometries can be described by such an explicit notation, rarely used
		\item implicit: typically implicit equations have more solutions than required and thus additional constraints have to be considered\\
		also rarely used in virtual reality
		\item parametric: the most common descriptive form\\
		no equivocations could occur, the definition is invariant to rotations\\
		infinite slopes can be described by tangential vectors which are finite\\
		typically consist of rational polynomials of the nth degree, most frequently cubic polynomials are used\\
		any point of the geometry can be exactly determined by mathematical functions (particular interest for CAD)
	\end{itemize}
	\item B\'ezier splines:
	\begin{itemize}
		\item first used in shipyards
		\item goal: interconnect given points by a smooth line
		\item points that approximate the curve are called "control points"
		\item mathematical function that delivers the result is named "base function", guarantees that the requirement for continuity in the control points is kept by interpolation or approximation
	\end{itemize}
	\item approximation: a curve approximates given control points / interpolation: the calculated curve has to meet the control points exactly; important in both cases that continuity is fulfilled
	\item "parametric continuity":
	\begin{itemize}
		\item represented by the letter C and its exponent i which defines the degree of the i\textsuperscript{th} derivation \\
		\includegraphics[scale=0.5]{3_24}
		\item C\textsuperscript{0} continuity: guarantees that the curve is not interrupted, but it could happen that a curve is not smooth in the point P1 but has an edge instead
		\item C\textsuperscript{1} continuity: the first derivation is continuous, all tangent vectors in the point P1 have the same slope, the curve is smooth at the point P1 (continuity of tangents)
		\item C\textsuperscript{2} continuity: the second derivation is continuous, the curve looks even smoother, also named as continuity of curvature
	\end{itemize}
\end{itemize}

\includegraphics[scale=0.6]{3_25}

\subsubsection*{Approximating Splines}

\begin{itemize}
	\item If an approximating curve is described by control points, there is an additional requirement that the resulting curve has to be within the polygon shaped by the control points
	\item The shape is a complex polygon that encloses all control points like a rubber band
	\item If the control points are extreme points, they are part of the polygon; otherwise, they are inside of the polygon
	\item The complex shape allows a good control of the curve (guarantees the curve is always within the visible area)\\
\includegraphics[scale=0.5]{3_26}
	\item B\'ezier Curve
	\begin{itemize}
		\item the base function is defined in a parametric form Q(t), t is a variable within [0, 1)
		\item the base function depends on the amount of control points; the whole curve is changed by just changing one point
		\item in addition to B\'ezier curve, also the first and last control point belong to the curve Q(t)
		\item always inside a complex polygon, because at least three control points are extreme points
		\item The Algorithm of de Casteljau: \\
		The base function of a B\'ezier curve can be calculated by the algorithm of de Casteljau\\
		idea: choose t and subdivide the first line b\textsubscript{0}b\textsubscript{1} at the distance t
		\begin{equation}
			b\textsubscript{0}\textsuperscript{1}(t) = (1-t)b\textsubscript{0}+t b\textsubscript{1}
		\end{equation}
		This process is repeated for the lines b\textsubscript{1}b\textsubscript{2} and b\textsubscript{2}b\textsubscript{3}, until in total three new points b\textsubscript{0}\textsuperscript{1}, b\textsubscript{1}\textsuperscript{1}, b\textsubscript{2}\textsuperscript{1} will result \\
		the calculation is repeated with the three new points and so on \\
		In general:
		\begin{equation}
			b\textsubscript{i}\textsuperscript{r}(t) = (1-t)b\textsubscript{i}\textsuperscript{r-1}+t b\textsubscript{i+1}\textsuperscript{r-1}
		\end{equation}
		\begin{equation}
			b\textsubscript{i}\textsuperscript{r}(t) = b\textsubscript{i}
		\end{equation}
		i = number of control points; i=0,...(n-r), r = number of lines; r=1,...,n
	\end{itemize}
	\includegraphics[scale=0.75]{3_29} \includegraphics[scale=0.75]{3_30}
	\begin{itemize}
		\item For a better approximation of the curve, more control points have to be used which will increase the computing time
		\item The algorithm of de Castelijau can also be used, if more than four control points exist
		\item The number of control points(L+1) defines the kind of curve
		\item For L=3 (cubic B\'ezier curve), it is further elaborated and simplified as:
		\begin{equation}
			Q(t) = b\textsubscript{0}(1-t)\textsuperscript{3}+b\textsubscript{1}3t(1-t)\textsuperscript{2}+b\textsubscript{2}3t\textsuperscript{2}(1-t)+b\textsubscript{3}t\textsuperscript{3}
		\end{equation}
		\item Bernstein Polynomials: \\
		B\'ezier curves can be more easily calculated by using it; do not work recursively but directly deliver the result at the position t by using polynomial coefficients \\
		When (L+1) control points are given, the function is defined by:
		\begin{equation}
			Q(t) = \sum_{i=0}^Lb\textsubscript{i} B\textsubscript{i}\textsuperscript{L}(t)
		\end{equation}
		$$ B\textsubscript{i}\textsuperscript{L}(t) $$ is the Bernstein polynomial, which is defined as:
		\begin{equation}
			B\textsubscript{i}\textsuperscript{L}(t)= \textsubscript{L}C\textsubscript{i} (1-t)\textsuperscript{L-i}t\textsuperscript{i}, \quad L >=i
		\end{equation}
	\end{itemize}
	\includegraphics[scale=0.65]{3_31}
	\begin{itemize}
		\item If B\'ezier curves of a high degree are employed, it is difficult to keep the curve smooth
		\item Bernstein polynomials can also be calculated by a successive superposition of linear B\'ezier splines
	\end{itemize}
	\includegraphics[scale=0.8]{3_32}
\end{itemize}

\subsubsection*{Interpolating Splines}

\begin{itemize}
	\item Boundary Representation - B-Rep
	\begin{itemize}
		\item Edge representations(or surface models or boundary representation) use 3D polygons to define the limiting surfaces of an object \\
		The limiting surfaces can be plain surfaces, but also surfaces of higher order \\
		It allows an exact mathematical description for many geometries, good approximation for the others
		\item B-Rep models consist of three object types: surfaces, edges, and corners
		\item To create on object out of the individual surfaces, the data structure must also contain a topological part(which defines the neighborhood of the surfaces) beside the normal geometric definition \\
		very common topological list contains 4 parts: a list of points/edges/surfaces/volumetric objects \\
		This list corresponds to a point-edge-surface model \\
		advantage: compact storage of all defining elements, since every point has to be stored only once
		\item The advantage of surface models consists in the complete availability of all topological information \\
		Surfaces, edges, and points can be stored as individual objects, which significantly increases the flexibility of the object \\
		one drawback: high calculation effort and complex network structure
	\end{itemize}
\end{itemize}

\includegraphics[scale=0.7]{3_56}

\subsubsection*{Volumetric Models}

\begin{itemize}
	\item The goal is to create objects that can be generally used (not only suitable for special applications)
	\item Only complete representations of physical objects are accepted
	\item Volumetric models can be distinguished into: \\
	cell models: these models discretize the required space of the object \\
	accumulative models: the object consists of a sum of basic elements \\
	generative objects: predefined volumetric elements (primitives) and Boolean operations on the elements \\
	other models: parametric models(e.g. CAD), sweep models, breakdown into elements(FEM)
	\item CSG(Constructive/Computational Solid Geometry)
	\begin{itemize}
		\item 3 primitive methods to shape geometry out of simple objects: merging parts(set union), cutting parts out(difference), calculate common parts of objects(intersection) $\rightarrow$ this approach is used by CSG
		\item CSG uses simple basic geometries like cones, spheres, cubes which are completely described by mathematical functions
		\item Complex objects can be described by a tree of Boolean operations		
	\end{itemize}
	\includegraphics[scale=0.8]{3_62}
	\item Cell Models
	\begin{itemize}
		\item In analogy to pixel models, cell models use voxels(volume elements) of uniform size to represent volumetric models
		\item the voxels are arranged in a regular 3D grid and are represented by the coordinates of the cell's center
		\item the maximum resolution is defined by the cell size
		\item the representation of 3D objects using cell models is suitable for the computation of volumes and other Boolean operations; however the geometrical elements(corner, edge, surface) can only be represented imprecisely or with a large effort
	\end{itemize}
\end{itemize}

\subsection{Color Definitions in the Computer}

\subsubsection*{Color Depth and Color Palettes}

\begin{itemize}
	\item The maximum number of colors to be used by the computer is defined by the color depth, i.e. the amount of bits that describe a color
	\item (number of colors) = $2^{color depth}$ (e.g. color depth 1 means 2 colors can be used)
	\item If the color depth is smaller than 8 bits, color palettes(color look-up tables, LUT) are used
	\begin{itemize}
		\item color palettes: an array in which each entry defines exactly one color
		\item thus, this array contains color information for each pixel
	\end{itemize}
	\item In the picture memory, references to the LUT exist
	\begin{itemize}
		\item if colors change, the entry in the LUT is modified while the picture memory stays unchanged
	\end{itemize}
	\item Each entry in the look-up table consists of a tuple of numbers for the colors RGB
	\item If the color depth is larger then 8 bits, the color look-up table would become too large and thus color information is directly stored in the picture memory
\end{itemize}

\includegraphics[scale=0.7]{3_64}

\subsubsection*{Color Quantization}

\begin{itemize}
	\item Color Quantization: to reduce the required memory for displaying images
	\begin{itemize}
		\item Due to the immense costs of memory chips, the available graphic memory was restricted for a long time
		\item Thus, old graphic cards could only display images with a color depth of 8 Bit with a poor resolution
		\item Images with a higher color depth had to be reduced to 8 Bit color depth
		\item Even today it is used e.g. graphics which are transferred over the Internet, are compressed in color to diminish the time for transmission
		\item Furthermore, human's eye can resolve a limited amount of different colors(color resolution from 5-8 Bits per color) so images can be reduced in color depth without being noticed by the user
	\end{itemize}
	\item Need for high color depth: typical images with a color gradient or antialiasing are only possible with a higher color depth
	\item The colors in the original image, which will be assigned to be the same color in the color palette, form a cluster(= color domain) \\
	\includegraphics[scale=0.4]{3_66}
	\item Pre-clustering procedure:
	\begin{itemize}
		\item divides the color space into a certain amount of clusters
		\item the color of a cluster is one color of the color palette
		\item RGB color space is uniformly divided into 256 colors without considering whether the colors exist in the real image or not
		\item this method delivers unsatisfying outcomes
	\end{itemize}
	\item Popularity algorithm:
	\begin{itemize}
		\item one of the algorithms to create an adapted color palette
		\item a histogram of colors is created: it is examined how often every color exists in the original image \\
		$\rightarrow$ A new color palette is generated out of m(=number of the desired color depth, e.g. 256) mostly used colors \\
		$\rightarrow$ the colors of the new palette are associated with the original colors of the image i.e. the colors of the palette which fit best the original color \\
		$\rightarrow$ the rest of the image colors is replaced by the one from the palette, which is very close to it (to determine the most similar color, the Euclidian distance is calculated)
		\item drawback:\\
		very time consuming; no structure is created so the complete color palette has to be searched for every color assignment \\
		any color details in small segments of the image could be completely wrong
	\end{itemize}
	\item Median-cut algorithm:
	\begin{itemize}
		\item In order to find the colors for the adapted color palette, the best approximation of the available colors is created instead of choosing colors from the original image
		\item all colors from the original image are marked in the RGB color cube \\
		$\rightarrow$ the cube is reduced in size until the cloud of marked colors fits the cube best \\
		$\rightarrow$ the cube is cut parallel to its longest edge(median-cut) in such a way that every sub-cube contains the same number of pixels \\
		$\rightarrow$  both sub-cubes are contracted until they fit exactly the number of pixels \\
		$\rightarrow$ again, the sub-cubes are cut along the median line and procedure recurs \\
		The procedure stops when a previously defined number of cub-cubes is reached
		\item in the best case, the resulting segments of the RGB color cube contain only one color that can be assigned to the pixels
		\item if more than one color is in the final cube, a mean value is calculated
	\end{itemize}
\end{itemize}

\includegraphics[scale=0.5]{3_69}

\begin{itemize}
	\item after the determination of the color palette, the difference(failure) between the needed(real) color and the associated color of the palette is not taken into account
	\item Floyd-Steinberg algorithm:
	\begin{itemize}
		\item in order to remove the remaining color steps, smoothens the color transitions
		\item the algorithm spreads this failure to the point of the neighborhood with the following weight: \\
		\includegraphics[scale=0.5]{3_71}
		\item the failures are gradually spread over the complete image without increasing the possible amount of palette colors
	\end{itemize}
\end{itemize}


\newpage
$$ \alpha \delta \Delta \delta \int_a^b f(x) \sum_{i=1}^n \text{exp}(-t) $$

\begin{equation}
	f(x) = \text{exp}(-t)
\end{equation}

\begin{align}
	A &= B \\
	&= C
\end{align}

\section{Chapter 5}

\end{document}
